\documentclass{article}
\usepackage{../../preamble}

\title{SINF1225 TP2}
\author{Groupe 12}
\date{February 2018}

\begin{document}

\maketitle

\section{Faits élémentaires}

\subsection{Relations unaires}

\begin{itemize}
    \item Utilisateur
        \begin{itemize}
        \item est identifié par (id) "JB007"
        \item a (pseudo) "JamesBond";
        \item a (password) "vesperlynd";
        \item a (mail) "james.bond@MI6.com";
        \item a (pic) "pdp/JB007.jpg";
        \item a (surname) "Bond";
        \item a (name) "James".
    \end{itemize}
    \item RelationAmitie
    \begin{itemize}
        \item concerne (id1) "JB007";
        \item concerne (id2) "VL069";
        \item est à l'état (status) "bestfriend";
        \item est initiée par (initiator) "JB007".
    \end{itemize}
    \item SondageAccord
    \begin{itemize}
        \item a (id) "JB007";
        \item a (pollid) "A123";
        \item a (question) "Date?";
        \item a (isclosed) "false".
    \end{itemize}
    \item PollParticipation
    \begin{itemize}
        \item se réfère à (pollid) "A123";
        \item concerne (id) "VL069".
    \end{itemize}
    \item Choix
    \begin{itemize}
        \item appartient à (pollid) "A123";
        \item propose (option) "Yes";
        \item concerne (id) "VB069".
    \end{itemize}
\end{itemize}

\subsection{Relations binaires}

\begin{itemize}
     \item Utilisateur
     \begin{itemize}
          \item est invité à (pollid) "A123";
     \end{itemize}
\end{itemize}

\subsection{Commentaires}

    \begin{itemize}
        \item La raison d'utiliser un identificateur est pour éviter de devoir revérifier tout quand un utilisateur change une des données. Avec un \textit{id}, qu'on ne peut pas changer, cela ne pose pas problème.

        C'est également valable pour les sondages. On prend des \textit{pollid} pour permettre de modéliser proprement les participants à un poll.
        \item
    \end{itemize}
\end{document}
