\documentclass{article}
\usepackage{../../preamble}

\title{SINF1225 TP2}
\author{Gilles Peiffer}
\date{February 2018}

\begin{document}

\maketitle

\section{Faits élémentaires}

\subsection{Relations unaires}

\begin{itemize}
    \item Utilisateur a
        \begin{itemize}
        \item (id) "JB007"
        \item (pseudo) "JamesBond";
        \item (password) "vesperlynd";
        \item (mail) "james.bond@MI6.com";
        \item (pic) "pdp/JB007.jpg";
        \item (surname) "Bond";
        \item (name) "James".
    \end{itemize}
    \item RelationAmi a
    \begin{itemize}
        \item (user1) "JB007";
        \item (user2) "VL069";
        \item (status) "bestfriend";
        \item (date) "01/01/1953";
        \item (initiator) "JB007".
    \end{itemize}
    \item SondageAccord a
    \begin{itemize}
        \item (author) "JB007";
        \item (pollid) "A123";
        \item (question) "Date?";
        \item (isclosed) "false";
    \end{itemize}
    \item PollRelation a
    \begin{itemize}
        \item (pollid) "A123";
        \item (participant) "VL069";
        \item (hasanswered) "false";
    \end{itemize}
    \item Choix a
    \begin{itemize}
        \item (pollid) "A123";
        \item (option) "Yes";
        \item (score) "0";
    \end{itemize}
\end{itemize}

\subsection{Commentaires}

    \begin{itemize}
        \item La raison d'utiliser un identificateur est pour éviter de devoir revérifier tout quand un utilisateur change une des données. Avec un \textit{id}, qu'on ne peut pas changer, cela ne pose pas problème.

        C'est également valable pour les sondages. On prend des \textit{pollid} pour permettre de modéliser proprement les participants à un poll.
        \item
    \end{itemize}
\end{document}
