\documentclass{article}
\usepackage{../../preamble}

\title{SINF1225 TP1}
\author{Gilles Peiffer}
\date{February 2018}

\begin{document}

\maketitle

\section{Premiers pas dans les bases de données}
\subsection{Insertions dans la BDD}

    Rempli à la main, en créant manuellement les colonnes (2 éléments rajoutés par table).

\subsection{Requêtes sur une seule table}

    \begin{lstlisting}[language=SQL]
    SELECT nom
    FROM Personne
    WHERE sex = 'M'; \end{lstlisting}

    Renvoie un tableau contenant 'Yoda' et 'Darth Maul'.

\subsubsection{Requêtes à faire}

    Sont repris ici uniquement les codes permettant d'arriver au résultat demandé.

\begin{enumerate}
    \item
    \begin{lstlisting}[language=SQL]
    SELECT ncli, nom, adresse, localite, cat, compte
    FROM Client
    WHERE localite = 'Namur'; \end{lstlisting}

    \item
    \begin{lstlisting}[language=SQL]
    SELECT ncli, nom, adresse, localite, cat, compte
    FROM Client
    WHERE localite = 'Lille' and compte >= 0; \end{lstlisting}

    \item
    \begin{lstlisting}[language=SQL]
    SELECT nom, adresse
    FROM Client
    WHERE cat = 'C1'; \end{lstlisting}

    \item
    \begin{lstlisting}[language=SQL]
    SELECT strftime('%Y', datecom)
    FROM Commande; \end{lstlisting}

    \item
    \begin{lstlisting}[language=SQL]
    SELECT npro, prix*qstock
    FROM Produit; \end{lstlisting}

    \item
    \begin{lstlisting}[language=SQL]
    SELECT sum(prix*qstock)
    FROM Produit; \end{lstlisting}
\end{enumerate}

\subsection{Requêtes sur plusieurs tables}
\subsubsection{Requêtes à faire}

    Sont repris ici uniquement les codes permettant d'arriver au résultat demandé.

\begin{enumerate}
    \item
    \begin{lstlisting}[language=SQL]
    SELECT DISTINCT C.nom, C.adresse, C.localite
    FROM Client C, Commande M
    WHERE C.ncli == M.ncli; \end{lstlisting}

    \item
    \begin{lstlisting}[language=SQL]
    SELECT P.npro, P.libelle, D.qcom
    FROM Client C, Commande M, Detail D, Produit P
    WHERE C.ncli == M.ncli and C.nom == 'Ferard' and D.ncom == M.ncom and D.npro = P.npro; \end{lstlisting}

    \item
    \begin{lstlisting}[language=SQL]
    SELECT P.npro, P.libelle, D.qcom, date(M.datecom)
    FROM Client C, Commande M, Detail D, Produit P
    WHERE C.ncli == M.ncli and C.nom == 'Ferard' and D.ncom == M.ncom and D.npro = P.npro; \end{lstlisting}

    \item
    \begin{lstlisting}[language=SQL]
    SELECT M.ncom, sum(D.qcom*P.prix)
    FROM Client C, Commande M, Detail D, Produit P
    WHERE C.ncli == M.ncli and C.nom == 'Ferard' and D.ncom == M.ncom and D.npro = P.npro
    GROUP BY M.ncom; \end{lstlisting}

    \item
    \begin{lstlisting}[language=SQL]
    SELECT sum(D.qcom*P.prix)
    FROM Client C, Commande M, Detail D, Produit P
    WHERE C.ncli == M.ncli and C.nom == 'Ferard' and D.ncom == M.ncom and D.npro = P.npro; \end{lstlisting}

    \item
    \begin{lstlisting}[language=SQL]
    SELECT C.nom, sum(D.qcom*P.prix)
    FROM Client C, Commande M, Detail D, Produit P
    WHERE C.ncli == M.ncli and D.ncom == M.ncom and D.npro = P.npro
    GROUP BY C.nom; \end{lstlisting}

    \item
    \begin{lstlisting}[language=SQL]
    SELECT C.nom, sum(D.qcom*P.prix)
    FROM Client C, Commande M, Detail D, Produit P
    WHERE C.ncli == M.ncli and D.ncom == M.ncom and D.npro = P.npro
    GROUP BY C.nom
    ORDER BY -sum(D.qcom*P.prix); \end{lstlisting}
\end{enumerate}

\subsection{Faire des changements dans la BDD}

    Sont repris ici uniquement les codes permettant d'arriver au résultat demandé.

\begin{enumerate}
    \item
    \begin{lstlisting}[language=SQL]
    ALTER TABLE Client
    ADD planete TEXT
    DEFAULT 'Terre'; \end{lstlisting}

    \item
    \begin{lstlisting}[language=SQL]
    UPDATE Client
    SET planete = 'Tatooine', nom = 'Skywalker'
    WHERE ncli == 'C123'; \end{lstlisting}

    \item
    \begin{lstlisting}[language=SQL]
    DELETE FROM Client
    WHERE planete == 'Tatooine'; \end{lstlisting}
\end{enumerate}
\end{document}
